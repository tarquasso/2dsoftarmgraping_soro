\section{Conclusion}
\label{sec:conclusion}
This work introduced a new manufacturing approach for a soft manipulator and demonstrated its capabilities through autonomously grasping-and-placing a randomly positioned object.
To produce the complete soft manipulator used, an entirely soft gripper was designed and fabricated and then combined with a previously developed soft robotic arm.
It was then shown that a minimal strain and collision-free approach to an object of interest can be achieved by posing the grasp motion plan as a series of constrained nonlinear optimization problems.

The fabrication approach presented has potential to generalize beyond just the fabrication of a gripper.
The new approach is advantageous because it allows for arbitrary designs of internal fluidic cavities and the casting of a homogeneous soft segment. It removes the need for laminating several separately casted parts together.
Such a homogeneous soft segment is less prone to rupture and to manufacturing inconsistencies, and would therefore allow for better robot performance.
x
The results presented in this paper suggest that despite their extreme compliance, soft robots are capable of repeatable object manipulation while simultaneously providing inherently safe interactions with their environment.
The manipulator is suitable to perform delicate tasks with low payloads, for example grasping objects that should not be squeezed and/or should not break during manipulation.
We demonstrated the ability of an entirely soft manipulator to autonomously grasp an object, which leads to many potential applications.
In a manufacturing setting, this could resemble a soft robot executing tasks requiring high dexterity when handling delicate objects.
In a human-centric environment, soft arm grasping manipulation may enable soft robots to interact safely with humans.
Furthermore, in a surgical setting, highly compliant soft robots with grippers may assist with operations in sensitive environments during tasks where no high level of precision is required.

