\documentclass[letterpaper, 10 pt, twocolumn, conference]{article}

\usepackage[paper=letterpaper,left=19.1mm,right=19.1mm,width=7in,top=19.1mm,bottom=19.1mm]{geometry}

\begin{document}

\title{SoRo Submission: Reply to Reviewer Comments}
\author{Robert Katschmann, Andrew Marchese and Daniela Rus}
\date{October 10th, 2015}
\maketitle

We thank both reviewers for their time and effort to provide us with thoughtful comments. We worked on each point raised by the reviewers and hope that we have a much clearer and improved manuscript. Below we detail how we addressed the reviewers’ concerns. The original reviewer comments are shown in italics, our answers are given underneath. 

%\clearpage

\section{Reviewer 1's Comments}

\textit{This paper discusses an autonomous object manipulation algorithm using a soft planar grasping manipulator.}

\subsection{Contribution Discussion}
\textit{Several major issues with the paper hold it back. First, the contribution of the paper seems very minor. A considerable amount of the paper is already covered in the work from [12-14], which involves the design of the robot, the recipe to create the robot, and the whole-arm motion planning. The contribution of this work is the grasping algorithm itself, which is a preplanned sequence of robot curvatures. In light of the whole arm motion planner in [12], this contribution seems minor when adding in a (circular) object.  The algorithm itself is implemented as a state-machine method, and with 4 or 5 states and seems trivial. Only one experimental test is performed to demonstrate the algorithm, and no quantitative error metrics are provided.}

We thank the reviewer for pointing out that the contribution is not clear. The contribution of this paper is an autonomous system for object pick-and-place manipulation that can handle objects of unknown geometry using only soft components to build the manipulator. The griper can pick up arbitrarily shaped objects (compatible with the size of the gripper) placed at arbitrary locations on the work space. 
TODO: We will revise the introduction section to make main contribution clear. We will also adjust the text to highlight the extensive experiments and the novelty. 

\textit{For these reasons and especially for the minor contribution that this manuscript offers, I cannot recommend publication as a journal paper in SoRo. The contribution seems better suited for a conference paper submission.}

We believe that the contribution of successfully and repeatedly performing object manipulation using a fully soft, multiple degree of freedom arm is a novel capability. Our feasibility study opens many avenues for soft robotic manipulation. We will revise the introduction to better articulate these ideas. 

\subsection{Motion-Capture}
\textit{With a motion-capture system, the problem becomes entirely well-defined.
However, soft robotics are specifically designed to work in confined or constrained environments where constraints and obstacles are unknown, with often no line of sight. Discuss how one might actually implement this without a motion-capture system that picks up the entire configuration of the robot manipulator.
Why does soft robotics systems need to specifically be designed for confined or constrained environments only? How is the use of motion capture systems a well-defined problem and how is it such especially in conjunction with soft robotics? Have others done it before, do you have a reference on this?}

We thank the reviewer for this observation. We used an external localization system in our experiments because localization was not the main point of the contribution. We developed an end-to-end system that can approximately locate an unknown object placed at an unknown location and move it somewhere else. The external localization system is a convenient way to approximately identify the location of the object and visualize/track how the object is moved in the space. We can replace the external localization system with any other method for localizing an object in the workspace. We will revise the exposition to make these points more clear. 

\subsection{Suitable contactless/constrained motion?}
\textit{Section 1.1 Last paragraph: “was only suitable for contactless motion” implies that your proposed method solved motion when the robot is constrained by the environment, but it is rather about gripping objects.}
TODO: Read that sentence again and try to understand what this comment means ?????

\subsection{Contribution 1}
\textit{Section 1.2 Contribution 1 seems entirely described in [12] and [13]. Contribution 4 seems obvious that force sensing nor accurate positioning is required to manipulate objects (as discussed in [21]).}

Contribution 1 is different than the contributions in 12 and 13, which focus on planning for the body of the robot. Here we plan for an enveloping grasp of an unknown object in the plane and demonstrate pick-and-place from uncertain locations. 

\subsection{System Overview Figure}
\textit{Section 2. System overview, Figure 2: The image is difficult to see and in its current form, and information of the physical frame surrounding the robot seems unnecessary.}

Thank you for this comment. We will revise the figure to make it clear what the system components are. We will be careful about the placement of labels. 

\subsection{Fabrication Steps Detailing}
\textit{Section 3.1-3.2 These sections are completely described in [13] and comes off as a recipe to make the manipulator, which is the contribution of [13], and therefore should not be reproduced in the main text. Figure 4 and Table 1. Pulled from [13] seem unnecessary. Furthermore Table 1 does not fit the column.}

Indeed, Reference 13 includes some description of the fabrication steps, but the process described in reference 13 is different; specifically 13 does not describe the fabrication of the manipulator in this paper. We revised the manuscript to discuss the points necessary to allow the reader to reproduce the system.

\subsection{Motion Tracker}
\textit{Section 4.2 par 1., Which motion tracker was used?}
We used the commercially available The Opti-Track System. We will revise the text to make this precise.

\subsection{Optimization Equation}
\textit{Section 4.2 The optimization equation, as I eventually found hidden in Algorithm 1, is difficult to understand: what are parameters v.s. variables, and what is R in the objective function? I suggest writing the equation in the text in full.}

We revised the manuscript to clarify this point. 

\subsection{Algorithm Difference}
\textit{I also find it difficult to see exactly the difference between the algorithm presented here and the one in [12]. While grasping is indeed is one objective, that seems to be realized by a curvature objective.} 

We thank the reviewer for this comment. We revised the exposition to highlight the differences between the two approaches. The algorithm in [12] is about planning the motion of a soft arm without a gripper through a maze at a centerline while taking the arms bulging shape as a trapezoidal into account. The approach does not work for approaching objects with the manipulator including a gripper, because a tip trajectory for successfully moving towards the object is not known, but needs to be generated. That trajectory needs to avoid pushing the object away with the manipulator trunk when approaching it with the soft gripper at its end. In this paper we present an algorithm for approaching an object by following along concentric lines, which are further and further decreasing in size until object size is achieved. The non-linear optimization is finding tangent poses along these lines using non-linear optimization and then follows down those concentric lines, which implicitly guarantees the arms pose to stay convex – since the circles are convex themselves.

\subsection{Optimization Constraints}
\textit{Section 4.2 Is the optimization function over or underconstrained? Will you always find a solution? What happens if you don’t find a solution?}

The optimization function is only over-constrained if we have to achieve a pose outside of the reachable workspace. Otherwise it is well-constrained. The system optimizes for the quadratic cost of having high curvatures. A solution is not found only if the target object pose is not within the reachable workspace of the arm, that means of the object was placed by the user outside of the workspace. The arm’s workspace can easily be calculated and added this formula to the paper. This workspace check is performed before doing any further calculations.

\subsection{Solver Details}
\textit{Section 4.2 par 1., Provide the speed of the optimization solver as well as the solver used.}

We included this information in the paper.

\subsection{Convexity as Constraint?}
\textit{Section 4.2 par 1. “…while its null space maintains a convex shape, bending away from the object”. Is the convexity actually a hard constraint on the solver? How much does this depend on your specific task and the orientation of the arm/gripper combo?}

When solving for a tangent gripper pose in CCW direction along a concentric circle, this is automatically the constraint for convexity.

\subsection{Algorithm Numbering}
\textit{Section 4.3 The text describing Algorithm 1 would be well suited to have numbered sections that could help the reader understand at which location in the Algorithm the text is referring to.}

We revised the manuscript accordingly. 

\subsection{Object Details}
\textit{Section 5.2 What is the object weight? What is the object radius?}

We added info about the sample objects we used in the paper.

\subsection{Task Execution Speed}
\textit{Figure 9: What is the time/speed of the task?}

We added this data to the paper. 

\subsection{Additional References}

\textit{Lack of references (Some covered in DeVolder2010):
De Volder, Michaël, and Dominiek Reynaerts. "Pneumatic and hydraulic microactuators: a review." Journal of Micromechanics and microengineering20.4 (2010): 043001.
Ikuta, Koji, Hironobu Ichikawa, and Katsuya Suzuki. "Safety-active catheter with multiple-segments driven by micro-hydraulic actuators." Medical Image Computing and Computer-Assisted Intervention—MICCAI 2002. Springer Berlin Heidelberg, 2002. 182-191.
J. Xiao and R. Vatcha, “Real-time adaptive motion planning for a continuum manipulator,” Proc. IROS 2010, pp. 5919–5926, Oct. 2010.
J. Li and J. Xiao, “A general formulation and approach to constrained, continuum manipulation,” Adv. Robot., no. July 2015, pp. 1–11, 2015.}

Thank you, we added these references to the paper.  

%\clearpage

\section{Reviewer 2's Comments}

\textit{This paper covers the fabrication and integration of a new soft robot gripper with a six degree of freedom soft multi-segment arm. The paper also describes a planner for grasping using the integrated system as well as some evaluation of grasping efficacy and the gripper workspace.}

\textit{In general, I think the topic is very interesting and the approach valuable. There seems to be a number of practical limitations to the platform and approach that I would like to see addressed or discussed in the paper. The following are specific feedback that I have:}

\subsection{Uncertainty Characterization}
\textit{For the third contribution listed on page 2 (line 33), I don't agree that you have characterized some of the uncertainty that is rather important. For example, you have shown for the gripper where it can grab for a single set of trials, but you do not describe repeating those trials, which would actually give you measures of uncertainty. Also, the red box used to denote the "bin" for placing the grasped object is an important measure of uncertainty. It seems like if this arm were to actually be used for manipulation, a good measure of uncertainty on positional accuracy would be important.}

Thank you for this comment, we revised the manuscript to highlight the contribution and provide the missing details on the experimentation. 

\subsection{Delicate Object Manipulation}
\textit{For the fourth contribution listed on page 2 (about line 37), I don't believe that you have shown (either experimentally or analytically) that you can manipulate delicate objects. I believe it, but you haven't reported gripper forces or extensive trials with delicate objects. I would either perform those tests or rewrite this contribution. It also isn't clear what is meant by "proper manipulation."}

Thank you for this observation, we added the missing details to the paper. 

\subsection{Clearer Fabrication Process Description}
\textit{I found the description of the fabrication process for the gripper to be slightly confusing. I think more annotations on figure 3 (such as point 1 or c, etc) and referring to those annotations in the description could make it clearer. Especially since this is listed as a major contribution.}

Figure 3’s point 1 and c were modified according to your suggestion and a note was added to the description in the paper.

\subsection{Distinguishing from Past Work}
\textit{Although you reference past papers from your own group, I think you could do a better job distinguishing from your own past work. Can the process for making the gripper be applied to the multi-segments as well? Which is better for what and why?}

We revised the exposition to clarify this point. 

\subsection{Limitations that could be Addressed}
\textit{Many limitations of the hardware or approach should be addressed in the paper. I realize that they cannot all be addressed or discussed. Nor am I asking for you to solve these problems, but an effort could be made to talk about the following:}
See our answers below.

\begin{enumerate}
\item \textit{How would the design of the gripper change if we wanted to grip in both directions? Is it even possible with the current design?}

Two grippers could be mounted next to each other to allow grasping in two directions

\item \textit{Is using motion tracking realistic for some of the scenarios you suggest? I would guess not, but what is the future for state estimation? Existing sensors? Or are new sensors needed?}

Exteroceptive sensing can be replaced by proprioceptive sensing using resistive bend sensors (Reference conference work IROS 2015)

\item \textit{It requires 6 degrees of freedom to achieve a reasonable reachable work space, how does this scale to a full 6 DoF task? Or does it scale? Is it limited to in-plane tasks? What is the reachable work space in the plane with 6 DoF and the + or - 60 degree joint limit?}

We revised the text to clarify this point. The focus here is planar manipulation. Scaling to a 3-dimensional task is not considered in this paper. Our current work considers the 3D case. 

\item \textit{Why was a convex shape necessary for approaching a grasp? What if I wanted to approach an object with a different orientation but in the same locations you already looked at?}

Thank you we revised the text to say that the shape needs not be convex. The convex shape approach is a conservative solution to guarantee minimal computation required in solving the planning problem while assuring not to collide. 

\item \textit{What is the role of the rollers in carrying your payload? Without the rollers, could the arm not move the object?}

The rollers minimize friction to the surface and therefor minimize frictional forces that need to be overcome. Stick-Slip Frction effects would be greatly increased if arm would be run directly on a surface with non-negligible friction coefficient. We added these details to the text.

\item \textit{Why did you decide to minimize manipulator deformation for your grasp object planner? Furthermore, what if I wanted my plan to follow the shortest distance for the end effector to travel (essentially following the black line in Figure 6), is this even possible given the possible kinematics of the arm?}

Minimized manipulator deformation is a feasible approach, because it is proportional to energy consumed by cylindrical piston drives and it minimizes strain to the actuators and minimizes risk of exerting further than actuation limit of a segment. We added this explanation to the paper. 

\end{enumerate}

\subsection{Grammar Comments}
\textit{Overall the paper is well-written, but there are some places with awkward or incorrect grammar. 
Examples include 
1)pg 2 line 37 "soft robots do neither require force sensing nor accurate ..."; 
2)pg 3 line 23 "Those seams are prone to rupture ..." refers to the laminated seams, but that isn't clear from phrasing; 
3)pg 7 line 21 "newly registers every single time the position of the placed object." is confusing.}

Thank you for pointing out these grammar mistakes. We proof read the paper and fixed them. 

\subsection{Video Attachement}
\textit{The paper should definitely include a video of operation of the arm. The overlaid figures are very well done, but video would be a valuable contribution to understanding the performance of the system, especially since we have no other time dependent graphs of end effector or joint position.}

Thank you for this suggestion, the video will be attached again to the revised version of the paper.

\subsection{Table 1 Formatting}
\textit{Table 1 needs to be formatted to stay in column}

We adjusted the formatting accordingly.

\subsection{Variable Definitions}
\textit{I feel like many variables could be more clearly defined. Things such as $L_{meas}$, $\phi$, $g_{off}$ (described in algorithm, but still not clear where measured from), $w_{off}$, $L_N$, $\kappa$ (in algorithm 1, not clear if current $\kappa$ or desired or ...), $\kappa_{off}$ (defined in algorithm, but not clear again what it was). Some of these terms are on the diagram in Figure 6, but their definition was still not clear to me. This was especially the case since I'm not sure what the multiple green circles on each concentric circle signified. Some terms were also used but not clearly such as "minimal tip transit distance".}

Thank you for pointing this out. We revised the text accordingly. We added a more explicit description of the variable names into the document. We described in greater detail the figures. 

\subsection{Value for $\delta d$}
\textit{Other items were clearly defined, but it wasn't clear how their value was set such as $\delta d$.}

We revised the text to describe how value for delta d was set

\subsection{Object Settling}
\textit{In Section 4.2, what does it mean for an "object to settle?"}

Object settling is defined that the object position has not changed within epsilon for T time. We added this detail to the manuscript.

\subsection{Forward Kinematics}
\textit{The forwKin procedure in algorithm 1 seemed a bit odd. It is recursive and requires calculating the forward kinematics of the previous link, all the way back to the base. That is fine, but the way it is defined, this would happen every time we step forward one link. Is that correct? Why not just use a for loop to be less confusing to a reader and more efficient computationally?}

We revised the text to clarify the forwKin procedure.

\subsection{Picking up Eggs}
\textit{Discussion of picking up eggs in section 5.1 is a bit out of place. Was actual testing done for this? Actually reporting grip forces or pressure would be much more interesting.}

Picking up eggs just as one example for delicate manipulation. We revised the text to make this clear. 

\subsection{Uncertainty of User Placement}
\textit{For the trial shown in Figure 7, what was the uncertainty on the user placement of the object? This seems like it could be rather large (compared to the resolution of the discrete placement locations) unless you used the motion capture system somehow.}

The placement by the user was accurate within +-2mm in relation to the discrete placement locations. This test mainly serves as a qualitative measure to show a relation between object size to gripper size to area of successful grasp. We added an explanation to the manuscript. 

\subsection{Final Drop Off Location Size}
\textit{How was an appropriate size determined for the red "bin" to determine success? It looks like the spread on placement was about 15 cm. That seems rather large and some commentary on it seems important.}

The red bin is a rectangle that was fit around the final motion after the experiments run to indicate the range wherein the objects were dropped. We added an explanation to the manuscript.

\subsection{Discussion Section in Paper}
\textit{The discussion in section 5.2 led me to have the following questions/comments:}
\begin{enumerate}
\item \textit{What if the object is not round?}

Picking up other objects that have a similar size compared to the round test object does also work. We included the other objects we used in the experiments in the manuscript. 

\item \textit{What is the importance of the rate that way points are sent from the planner? Especially failure trials where the arm went unstable seemed to indicate that both the controller and planner may be very dependent on smoothing the desired curvature way points or sending them slowly in time. More detail would be good.}

We revised the manuscript to provide more detail. Specifically, a new waypoint is sent to the controller immediately after arriving within delta epsilon of the previous waypoint, the controllers for each arm segment then compensate for the new delta in angle as quickly as possible to the new pose. That is the PD controller. Smoothing of the trajectory with several intermediate waypoints was found to be necessary. For the given workspace of the arm, we found X intermediate waypoints to be working solution.

\item \textit{ The instability in general seems important and would be nice if it was determined if it was from the planner or controller.}

The instability was identified to come from stick slip friction to the ground, given a roller ball contact that is not continuously smooth. If a segment gets a little bit stuck on the ground, the moment it releases, the soft body releases like a spring and can cause an overshoot, which the controller has to again compensate. Another source of failure is when a marker along the arm gets occluded or merges with an adjacent marker, causing the tracking system loose track of a measured arm segment temporarily and therefore causing the control loop to not close properly. We added a description to the manuscript. 
\end{enumerate}

\subsection{High Dexterity Observation}
\textit{Discussion in conclusion refers to high dexterity when handling delicate objects. However, all of your tests showed approaching the object from more or less the same direction. Do you have a kinematic model or experimental data that shows your manipulator's dexterous workspace? This is similar to a previous comment above.}

Thank you for this observation. We revised the conclusions to remove the high dexterity discussion; instead we describe how widely the arm can be stretched in its various poses to approach the same object. 


\end{document}